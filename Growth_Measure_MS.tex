\documentclass[]{article}
\usepackage{lmodern}
\usepackage{amssymb,amsmath}
\usepackage{ifxetex,ifluatex}
\usepackage{fixltx2e} % provides \textsubscript
\ifnum 0\ifxetex 1\fi\ifluatex 1\fi=0 % if pdftex
  \usepackage[T1]{fontenc}
  \usepackage[utf8]{inputenc}
\else % if luatex or xelatex
  \ifxetex
    \usepackage{mathspec}
  \else
    \usepackage{fontspec}
  \fi
  \defaultfontfeatures{Ligatures=TeX,Scale=MatchLowercase}
\fi
% use upquote if available, for straight quotes in verbatim environments
\IfFileExists{upquote.sty}{\usepackage{upquote}}{}
% use microtype if available
\IfFileExists{microtype.sty}{%
\usepackage{microtype}
\UseMicrotypeSet[protrusion]{basicmath} % disable protrusion for tt fonts
}{}
\usepackage[margin=1in]{geometry}
\usepackage{hyperref}
\hypersetup{unicode=true,
            pdftitle={RGR MS},
            pdfauthor={Chris H. Wilson},
            pdfborder={0 0 0},
            breaklinks=true}
\urlstyle{same}  % don't use monospace font for urls
\usepackage{graphicx,grffile}
\makeatletter
\def\maxwidth{\ifdim\Gin@nat@width>\linewidth\linewidth\else\Gin@nat@width\fi}
\def\maxheight{\ifdim\Gin@nat@height>\textheight\textheight\else\Gin@nat@height\fi}
\makeatother
% Scale images if necessary, so that they will not overflow the page
% margins by default, and it is still possible to overwrite the defaults
% using explicit options in \includegraphics[width, height, ...]{}
\setkeys{Gin}{width=\maxwidth,height=\maxheight,keepaspectratio}
\IfFileExists{parskip.sty}{%
\usepackage{parskip}
}{% else
\setlength{\parindent}{0pt}
\setlength{\parskip}{6pt plus 2pt minus 1pt}
}
\setlength{\emergencystretch}{3em}  % prevent overfull lines
\providecommand{\tightlist}{%
  \setlength{\itemsep}{0pt}\setlength{\parskip}{0pt}}
\setcounter{secnumdepth}{0}
% Redefines (sub)paragraphs to behave more like sections
\ifx\paragraph\undefined\else
\let\oldparagraph\paragraph
\renewcommand{\paragraph}[1]{\oldparagraph{#1}\mbox{}}
\fi
\ifx\subparagraph\undefined\else
\let\oldsubparagraph\subparagraph
\renewcommand{\subparagraph}[1]{\oldsubparagraph{#1}\mbox{}}
\fi

%%% Use protect on footnotes to avoid problems with footnotes in titles
\let\rmarkdownfootnote\footnote%
\def\footnote{\protect\rmarkdownfootnote}

%%% Change title format to be more compact
\usepackage{titling}

% Create subtitle command for use in maketitle
\newcommand{\subtitle}[1]{
  \posttitle{
    \begin{center}\large#1\end{center}
    }
}

\setlength{\droptitle}{-2em}

  \title{RGR MS}
    \pretitle{\vspace{\droptitle}\centering\huge}
  \posttitle{\par}
    \author{Chris H. Wilson}
    \preauthor{\centering\large\emph}
  \postauthor{\par}
      \predate{\centering\large\emph}
  \postdate{\par}
    \date{March 2, 2019}


\begin{document}
\maketitle

\subsubsection{Key Points}\label{key-points}

\begin{enumerate}
\def\labelenumi{\arabic{enumi})}
\item
  Terminology: ``linear difference'' -\textgreater{} ``linear growth'';
  ``log measure'' -\textgreater{} ``log RGR''; ``linear RGR != linear
  growth''
\item
  Be clear about the dimensional comparisons. The RGR measures are not
  dimensionally the same as linear growth (AGR).
\item
  RGR measures and all comparisons really, sensitive to the delta
  (elapsed time between measurements). Linear RGR \textgreater{} log RGR
  for much of the curve.
\item
  Linear growth (AGR) is just as good as log RGR or better for much of
  the curve in terms of dynamics/prediction.
\item
  If the goal is to estimate a treatment effect, linear growth at least
  is more interpretable, and subject to less statistical estimation
  variability.
\item
  Ulimately, things are inherently complicated in this data limited
  setting. The problem is when we do NOT know WHERE on the growth curve
  our observations are coming from. None of the measures discussed here
  get around this unavoidable fact.
\item
  No reason to insist on log RGR in this setting; in fact, several
  reasons to NOT use it.
\end{enumerate}

\subsubsection{Background and Rationale}\label{background-and-rationale}

Analyzing growth of individuals is fundamental in many areas of ecology
and biology. A common situation is the need to compare multiple
individuals across genotypes or species in experimental or observational
settings where variations in initial sizes and environmental factors
both contribute to observed variations in growth. In this setting, a
common default practice is to re-express growth as a relative measure,
dividing the growth increment by the initial size. In the limit as the
time period goes to zero, this can be represented as
\[\frac{dS}{dt}\frac{1}{S}\]

Without explicit specification of a time-varying dynamic, e.g.~some kind
of non-linear growth function, this representation corresponds to
exponential growth. That is, the quantity obtained by integration of
\[\frac{dS}{dt}\frac{1}{S} = k\] over some time period, and given some
initial size \(S_0\)

is simply the familiar exponential equation \[S_t = S_0e^{kt}\]

The quantity in equation 1 is often referred to as relative growth rate
(RGR), and the usual method of quantification, hereafter the ``log RGR''
corresponds to the solution in 2, as is readily checked. The log RGR is,
simply \(\frac{log(S_2) - log(S_1)}{t}\) The log RGR is very frequently
utilized as a default in place of taking the difference
\(\frac{S_2 - S_1}{t}\), hereafter ``linear growth rate'', since it is
seen as more effectively accounting for variations in initial size,
given the understanding that size itself is a fundamental driver of
subsequent growth. Note that the linear growth rate could also be
divided by \(S_1\) to return a ``linear RGR''.

XXX et al. (2012) summarized several flaws of the log RGR and
recommended instead to fit non-linear growth functions. The non-linear
growth functions can then be differentiated with respect to size \(S\)
in order to obtain a superior RGR measure. I wholeheartedly concur with
this advice. However, ecologists are often confronted with datasets
where only 2 or 3 time periods are available, thus precluding effective
fitting of non-linear functions. In this context, the log RGR is widely
recommended.

In this note, I demonstrate that the linear measures (linear growth
rate, and linear RGR) are in many cases superior to the log RGR. To be
sure, these quantities should be seen as answers to subtley different
questions. My intent is to highlight the underlying assumptions, and
challenge the status of log RGR as a default in the data limited
setting.

\subsubsection{Conceptual Overview}\label{conceptual-overview}

First, we assume a basic theoretical framework for growth: the sigmoidal
curve. Nearly every biologically motivated growth model follows
sigmoidal behavior. For instance, West et al. (2001) famously derived a
sigmoidal equation for growth from metabolic scaling theory. Mechanistic
models of photosynthesis and leaf area also result in sigmoidal growth
(CITE). Although particular sigmoidal models can be challenged, the
qualitative pattern is universal. Therefore, we will compare log RGR and
linear growth rate to a sigmoidal curve, which is itself presumed to
better approximate underlying biological/ecological reality. The
conceptual basis of my analysis is encapsulated in the following figure:

\begin{figure}
\includegraphics[width=1\linewidth]{Conceptual_Fig} \caption{Conceptual Argument}\label{fig:unnamed-chunk-2}
\end{figure}

The first portion of the sigmoidal curve is convex. In this zone, log
RGR is a decent approximation to underyling growth. However, there is an
adjcent, larger zone of approximate linearity, where the linear growth
rate describes biological reality more closely. Finally, neither
approximation is great in the upper portion of the curve, well into its
concave portion, although the linear approximation is uniformly superior
throughout the concave zone.

\subsubsection{Mathematical Analysis of Log
RGR}\label{mathematical-analysis-of-log-rgr}

\subsection{General Taylor Series
Approximation}\label{general-taylor-series-approximation}

Mathematically, the argument can be boiled down for any generic equation
for growth over time: \(S_t = f(S,t)\). Using Taylor Series, we can
approximate around some value \(a\) to second order with a generic
function \(g(S,t)\) as:
\[f(S,t) \approx g(S=a,t) + g'(S=a,t) \times (S-a) + \frac{1}{2}g''(S=a,t)\times(S-a)^2\]

As noted above, the canonical log RGR corresponds to exponential
dynamics \(S_t = S_0e^{kt}\). Use of exponential dynamics to approximate
\(S_t = f(S,t)\) obviously only works well where both the first and
second derivative of \(f(S,t)\) are positive (i.e.~where function is
convex). Given that the second derivative of any sigmoidal curve flips
from positive to negative, this approximation error grows rapidly
outside of a narrow zone.

\subsection{Analysis}\label{analysis}

We take the familiar logistic equation as a reasonable representaton for
sigmoidal growth, while noting that many options are available:

\[S_t = \frac{KS_0e^{rt}}{K+S_0(e^{rt}-1)}\]

Given this representation of underlying growth, the first question is:
what does the canonical log measure \(\frac{log(S_2) - log(S_1)}{t}\)
correspond to? In other words, we want to ask what happens given exact
measurements of \(S_2\) and \(S_1\), given that they are sampled from
above equation. For \(log(S_t)\) We have:

\[\log(S_t) = log(K) + log(S_0) + rt - log(K + S_0(e^{rt}-1))\]

If we have size observations \(S_1\) and \(S_2\) from two times, \(t_1\)
and \(t_2\),the difference between them is:

\[\log(S_2) - log(S_1) = r(t_2 - t_1) + log(\frac{K + S_0(e^{rt_1}-1)}{K + S_0(e^{rt_2}-1)})\]

For any given interval \(t_2~t_1\):\\
\[\frac{log(S_t) - log(S_0)}{t_2 - t_1} = r + \frac{1}{t_2-t_1}log(\frac{K + S_0(e^{rt_1}-1)}{K + S_0(e^{rt_2}-1)})\]
One flaw of log RGR (as pointed out previously by XXXX) is that RGR is
really time-varying, but in effect treated as though time constant (by
necessity given the limitation of data). If we want to investigate how
this quantity varies with sampling of arbitrary timepoints \(t_2\) and
\(t_1\) along the sigmoidal curve, we re-express \(t_2=t_1+\Delta\) and
let \(t_1=t\)
\[\frac{log(S_{t+\Delta}) - log(S_t)}{\Delta} = r + \frac{1}{\Delta}log(\frac{K + S_0(e^{rt}-1)}{K + S_0(e^{r({t+\Delta})}-1)})\]

\subsubsection{Log RGR versus Linear
RGR}\label{log-rgr-versus-linear-rgr}

Graphically, the comparison with the observed growth increments is:
\includegraphics{Growth_Measure_MS_files/figure-latex/unnamed-chunk-3-1.pdf}

Although striking, this comparison is misleading since the quantites
differ in dimesion. The log measure is really a \textbf{rate} with
dimensions of \(\frac{1}{time}\), and I should note that the usual
re-expression as \(g~g^-1~time^-1\) is unhelpful at best. In order to
make a dimensionally valid comparison, the linear measure must be
divided by the initial mass. Here is the result of doing so:

\includegraphics{Growth_Measure_MS_files/figure-latex/unnamed-chunk-4-1.pdf}

As expected, the log RGR outperforms the linear RGR in matching the real
(time varying) RGR early in the convex portion of the curve, then gets
outperfomed by the linear RGR thereafter. Both become pretty bad in the
concave portion of the curve.

The question is, given a set of data with one or two observation
periods, what quantity should be analyzed? The log RGR, the linear
growth rate, or perhaps the linear RGR? First, as noted above, these
quantities are answers to different questions at some level. The RGR
measures differ in dimension from the linear growth rate measure. They
answer the question: how much new growth occurs as a function of size?
The linear growth rate extrapolates growth independent of size. Despite
the conceptual defects of the latter,\\
\textbf{the only reason to prefer the former is the idea that, in the
long run, it will better predict growth dynamics.}

We have already rejected the idea that, in the data limited setting
considered here, we are getting an accurate understanding of
time-varying growth. In order to game this out, we need to look at
implied dynamics.

\subsubsection{Comparing log RGR and linear growth in terms of
dynamics}\label{comparing-log-rgr-and-linear-growth-in-terms-of-dynamics}

Use of the linear growth rate \(\frac{S_2 - S_1}{t}\) corresponds to
assumption of a static linear growth rate dynamic, just as use of
\(\frac{log(S_2) - log(S_1)}{t}\) corresponds to assuming a static
exponential growth rate dynamic. In the latter case, the log RGR has the
nice property of representing an ergodic observable (sensu Peters and
Gell-man 2016), but is only truly valid assuming exponential growth.
While widely (and rightly) dismissed as unrealistic, the linear growth
rate \(\frac{S_2 - S_1}{t}\) may in fact be a generally superior measure
for ecological analysis where no time series of size/biomass data is
available.

The comparison made here is the goodness of fit implied by replacing the
sigmoidal \(S_t\) with either a linear approximation or an exponential
approximation, given sampling of size from two pairs of time points: 1)
from the early (``exponential'') portion of sigmoid curve, and 2) from
the middle (``linear'') portion of sigmoidal curve, and 3) from the
saturating part of curve.

\begin{enumerate}
\def\labelenumi{\arabic{enumi})}
\tightlist
\item
  \(t_2 = 14\) and \(t_1=2\):
\end{enumerate}

\begin{verbatim}
## Warning: Removed 33 rows containing missing values (geom_path).
\end{verbatim}

\includegraphics{Growth_Measure_MS_files/figure-latex/unnamed-chunk-5-1.pdf}

As expected, the exponential approximation works better with data from
the convex portion of growth curve. However, the improvement is marginal
in absolute value, and quickly diverges outside of the convex portion
(in accord with our intuitive model).

\begin{enumerate}
\def\labelenumi{\arabic{enumi})}
\setcounter{enumi}{1}
\tightlist
\item
  \(t_2 = 25\) and \(t_1 = 12\):
\end{enumerate}

\begin{verbatim}
## Warning: Removed 18 rows containing missing values (geom_path).
\end{verbatim}

\begin{verbatim}
## Warning: Removed 2 rows containing missing values (geom_path).
\end{verbatim}

\includegraphics{Growth_Measure_MS_files/figure-latex/unnamed-chunk-6-1.pdf}

As can be seen, the linear model is a better approximation where data
are taken from within the center part of the growth cycle. Again, the
improvement is marginal, but real. Forecast accuracy is much higher, and
backcast accuracy marginally worse.

\begin{enumerate}
\def\labelenumi{\arabic{enumi})}
\setcounter{enumi}{2}
\tightlist
\item
  \(t_2=45\) and \(t_1=30\)
\end{enumerate}

\includegraphics{Growth_Measure_MS_files/figure-latex/unnamed-chunk-7-1.pdf}

As expected, in this scenario, the linear approximation is uniformly
better and thus always to be preferred.

At this point, we have seen that the linear growth rate has a large
range in which it is superior to the log RGR in terms of implied
dynamics (ability to forecast real growth), as well as in terms of
approximating the ``true'' instantaneous RGR.

\subsubsection{Statistical Properties of Log
RGR}\label{statistical-properties-of-log-rgr}

We can derive the sampling distribution of the log RGR
\(\frac{log(S_2) - log(S_1)}{t}\) based on a Taylor Series'
approximation. Specifically, we consider measurements \(S_2\) and
\(S_1\) with normally distributed error, where the variance scales with
the mean (a fairly typical property in biological/ecological data). The
distribution of \(S_2\) - \(S_1\) is then simply the difference of two
Normals. Next, we approximate the moments of the distribution of
\(log(S_t)\) as:
\[E[log(S_t)] \approx log(\mu_{S_t}) - \frac{\sigma^2}{2*\mu_{S_t}^2}\].
Using the delta method for variance, we have:
\[Var[log(S_t)] \approx \frac{\sigma^2}{\mu_{S_t}^2}\]

Given a constant coefficient of variation (reflecting variance scaling
with mean on the original scale) \(CV_s\) the sampling distribution of
\(\frac{log(S_2) - log(S_1)}{t}\) (setting \(t\) to unit scale), is
therefore: \[ N(log(S_2) - log(S_1), \sqrt2CV_s)\]

The CV of the log measure is then related to the expectaiton of the Z
scores of the new sampling distribution, and is inversely proportional
to statistical power: \[CV_l =\frac{\sqrt2}{log(S_2) - log(S_1)}CV_s\]

Thus, where \(log(S_2) - log(S_1) > \sqrt2\) over a unit time increment,
the log RGR should have greater statistical power, while it loses
statistical power as the log RGR declines. This can be reformulated as
\(S_2=S_1e^{\sqrt2}\), revealing a \emph{scale free} property of
statistical analysis of the log RGR. Specifically, wherever the
multiplicative growth increase on a unit time scale \(<e^{\sqrt2}\), the
log RGR has worse statistical properties than the linear growth rate.

For our previous parameter value simulations above, here is the curve of
\(CV_l\) with time \(t\), expressed in multiples of \(CV_s\):
\includegraphics{Growth_Measure_MS_files/figure-latex/unnamed-chunk-8-1.pdf}

As can be seen, in this situation, it is always worse! What happens if
we accelerate growth rate considerably (10X)

\includegraphics{Growth_Measure_MS_files/figure-latex/unnamed-chunk-9-1.pdf}

Expressed on a dimensionless time scale representing multiples of
\(\Delta\) we see that there is a small zone of equivalence. But
essentially, it is uniformly less powerful. In the end, it is perhaps
not too surprising that it is more difficult to estimate an RGR than a
GR. Nevertheless, this should give pause to default use of log RGR.

\subsubsection{Case Study}\label{case-study}

Maybe not necessary here. Theoretical demonstration feels complete.
OTOH, data are what motivate this!

\subsubsection{Conclusions and
Recommendations}\label{conclusions-and-recommendations}

In summary, the chief virtue of the log RGR
\(\frac{log(S_2) - log(S_1)}{t}\) measure in the data limited setting is
that it is best suited to estimate RGR in approximately exponential
phases of growth. In a narrow band of the sigmoid curve, it is arguably
superior to working with linear growth rate or RGR.

The much maligned linear growth rate is a superior default on three
grounds therefore. First, it represents an intuitive quantity with
dimensions of size/length/mass, whereas any RGR has dimensions of
\(\frac{1}{time}\)). Second, interpreted in terms of dynamics, it is
\emph{a better approximation} than the exponential dynamics implied by
the log RGR in many parts of the growth curve. Third, it has superior
statistical properties almost everywhere.

The widespread use of the log RGR \(\frac{log(S_2) - log(S_1)}{t}\) as
an \emph{a priori} preferred default for the data-limited situation
should be abandoned. Where only two or three time points are available,
fitting a linear growth trend is just as good if not better than
estimating an exponential growth rate (log RGR). Unfortunately, none of
the measures discussed here can overcome a critical problem in this
setting: we do NOT know what portion of the sigmoidal growth curve we
are sampling from, in many cases. In this context, careful thought and
attention is needed in making comparisons among possible measures.

The ideal scenario is to collect a proper time series (5-7+) and fit a
proper growth model. Where data are at all limiting, I recommend careful
incorporation of literature values and other external information as
priors in a fully Bayesian analysis in order to regularize inferences.


\end{document}
