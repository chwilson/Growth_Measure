\documentclass[]{article}
\usepackage{lmodern}
\usepackage{amssymb,amsmath}
\usepackage{ifxetex,ifluatex}
\usepackage{fixltx2e} % provides \textsubscript
\ifnum 0\ifxetex 1\fi\ifluatex 1\fi=0 % if pdftex
  \usepackage[T1]{fontenc}
  \usepackage[utf8]{inputenc}
\else % if luatex or xelatex
  \ifxetex
    \usepackage{mathspec}
  \else
    \usepackage{fontspec}
  \fi
  \defaultfontfeatures{Ligatures=TeX,Scale=MatchLowercase}
\fi
% use upquote if available, for straight quotes in verbatim environments
\IfFileExists{upquote.sty}{\usepackage{upquote}}{}
% use microtype if available
\IfFileExists{microtype.sty}{%
\usepackage{microtype}
\UseMicrotypeSet[protrusion]{basicmath} % disable protrusion for tt fonts
}{}
\usepackage[margin=1in]{geometry}
\usepackage{hyperref}
\hypersetup{unicode=true,
            pdftitle={RGR MS},
            pdfauthor={Chris H. Wilson},
            pdfborder={0 0 0},
            breaklinks=true}
\urlstyle{same}  % don't use monospace font for urls
\usepackage{color}
\usepackage{fancyvrb}
\newcommand{\VerbBar}{|}
\newcommand{\VERB}{\Verb[commandchars=\\\{\}]}
\DefineVerbatimEnvironment{Highlighting}{Verbatim}{commandchars=\\\{\}}
% Add ',fontsize=\small' for more characters per line
\usepackage{framed}
\definecolor{shadecolor}{RGB}{248,248,248}
\newenvironment{Shaded}{\begin{snugshade}}{\end{snugshade}}
\newcommand{\KeywordTok}[1]{\textcolor[rgb]{0.13,0.29,0.53}{\textbf{#1}}}
\newcommand{\DataTypeTok}[1]{\textcolor[rgb]{0.13,0.29,0.53}{#1}}
\newcommand{\DecValTok}[1]{\textcolor[rgb]{0.00,0.00,0.81}{#1}}
\newcommand{\BaseNTok}[1]{\textcolor[rgb]{0.00,0.00,0.81}{#1}}
\newcommand{\FloatTok}[1]{\textcolor[rgb]{0.00,0.00,0.81}{#1}}
\newcommand{\ConstantTok}[1]{\textcolor[rgb]{0.00,0.00,0.00}{#1}}
\newcommand{\CharTok}[1]{\textcolor[rgb]{0.31,0.60,0.02}{#1}}
\newcommand{\SpecialCharTok}[1]{\textcolor[rgb]{0.00,0.00,0.00}{#1}}
\newcommand{\StringTok}[1]{\textcolor[rgb]{0.31,0.60,0.02}{#1}}
\newcommand{\VerbatimStringTok}[1]{\textcolor[rgb]{0.31,0.60,0.02}{#1}}
\newcommand{\SpecialStringTok}[1]{\textcolor[rgb]{0.31,0.60,0.02}{#1}}
\newcommand{\ImportTok}[1]{#1}
\newcommand{\CommentTok}[1]{\textcolor[rgb]{0.56,0.35,0.01}{\textit{#1}}}
\newcommand{\DocumentationTok}[1]{\textcolor[rgb]{0.56,0.35,0.01}{\textbf{\textit{#1}}}}
\newcommand{\AnnotationTok}[1]{\textcolor[rgb]{0.56,0.35,0.01}{\textbf{\textit{#1}}}}
\newcommand{\CommentVarTok}[1]{\textcolor[rgb]{0.56,0.35,0.01}{\textbf{\textit{#1}}}}
\newcommand{\OtherTok}[1]{\textcolor[rgb]{0.56,0.35,0.01}{#1}}
\newcommand{\FunctionTok}[1]{\textcolor[rgb]{0.00,0.00,0.00}{#1}}
\newcommand{\VariableTok}[1]{\textcolor[rgb]{0.00,0.00,0.00}{#1}}
\newcommand{\ControlFlowTok}[1]{\textcolor[rgb]{0.13,0.29,0.53}{\textbf{#1}}}
\newcommand{\OperatorTok}[1]{\textcolor[rgb]{0.81,0.36,0.00}{\textbf{#1}}}
\newcommand{\BuiltInTok}[1]{#1}
\newcommand{\ExtensionTok}[1]{#1}
\newcommand{\PreprocessorTok}[1]{\textcolor[rgb]{0.56,0.35,0.01}{\textit{#1}}}
\newcommand{\AttributeTok}[1]{\textcolor[rgb]{0.77,0.63,0.00}{#1}}
\newcommand{\RegionMarkerTok}[1]{#1}
\newcommand{\InformationTok}[1]{\textcolor[rgb]{0.56,0.35,0.01}{\textbf{\textit{#1}}}}
\newcommand{\WarningTok}[1]{\textcolor[rgb]{0.56,0.35,0.01}{\textbf{\textit{#1}}}}
\newcommand{\AlertTok}[1]{\textcolor[rgb]{0.94,0.16,0.16}{#1}}
\newcommand{\ErrorTok}[1]{\textcolor[rgb]{0.64,0.00,0.00}{\textbf{#1}}}
\newcommand{\NormalTok}[1]{#1}
\usepackage{graphicx,grffile}
\makeatletter
\def\maxwidth{\ifdim\Gin@nat@width>\linewidth\linewidth\else\Gin@nat@width\fi}
\def\maxheight{\ifdim\Gin@nat@height>\textheight\textheight\else\Gin@nat@height\fi}
\makeatother
% Scale images if necessary, so that they will not overflow the page
% margins by default, and it is still possible to overwrite the defaults
% using explicit options in \includegraphics[width, height, ...]{}
\setkeys{Gin}{width=\maxwidth,height=\maxheight,keepaspectratio}
\IfFileExists{parskip.sty}{%
\usepackage{parskip}
}{% else
\setlength{\parindent}{0pt}
\setlength{\parskip}{6pt plus 2pt minus 1pt}
}
\setlength{\emergencystretch}{3em}  % prevent overfull lines
\providecommand{\tightlist}{%
  \setlength{\itemsep}{0pt}\setlength{\parskip}{0pt}}
\setcounter{secnumdepth}{0}
% Redefines (sub)paragraphs to behave more like sections
\ifx\paragraph\undefined\else
\let\oldparagraph\paragraph
\renewcommand{\paragraph}[1]{\oldparagraph{#1}\mbox{}}
\fi
\ifx\subparagraph\undefined\else
\let\oldsubparagraph\subparagraph
\renewcommand{\subparagraph}[1]{\oldsubparagraph{#1}\mbox{}}
\fi

%%% Use protect on footnotes to avoid problems with footnotes in titles
\let\rmarkdownfootnote\footnote%
\def\footnote{\protect\rmarkdownfootnote}

%%% Change title format to be more compact
\usepackage{titling}

% Create subtitle command for use in maketitle
\newcommand{\subtitle}[1]{
  \posttitle{
    \begin{center}\large#1\end{center}
    }
}

\setlength{\droptitle}{-2em}
  \title{RGR MS}
  \pretitle{\vspace{\droptitle}\centering\huge}
  \posttitle{\par}
  \author{Chris H. Wilson}
  \preauthor{\centering\large\emph}
  \postauthor{\par}
  \predate{\centering\large\emph}
  \postdate{\par}
  \date{March 2, 2019}


\begin{document}
\maketitle

\[\S_t = \frac{KS_0e^{rt}}{K+P_0(e^{rt}-1)}\]

Now, if we log transform our data, as practiced when calculating RGR
using the canonical approach, our functional is:

\[\log(S_t) = log(K) + log(S_0) + rt - log(K + P_0(e^{rt}-1))\]

If we have size observations \(S_1\) and \(S_2\) from two times, \(t_1\)
and \(t_2\),the difference between them is:

\[\log(S_2) - log(S_1) = r(t_2 - t_1) + log(\frac{K + P_0(e^{rt_1}-1)}{K + P_0(e^{rt_2}-1)})\]
If we simplify, and set \(t_1=0\), re-expressing \(t_2=t\), we have:

\[\log(S_2) - log(S_1) = rt + log(\frac{K}{K + P_0(e^{rt}-1)})\]

Thus, if we theoretically make error-free observations over time of a
plant, or any other organism for that matter, following sigmoidal growth
function above, and re-express our growth increment as the log of the
size at time \(t\) minus log of initial size, our observations will
follow this curve:

\includegraphics{Growth_Measure_MS_files/figure-latex/sigmoid log-sigmoid comparison-1.pdf}

Thus, a good deal of the potential for growth is washed out of the
measure, particularly of course where the sigmoidal curve flips from
convex to concave. This is not theoretically surprising, and, as noted
above, we concur with the general recommendation to always fit a more
suitable non-linear growth model where a time series of data
\({S}^N_{i=1}\) is available. The practical point here is that use of
the log-transformed measure should \emph{not} be routine in analyses of
ecological growth data where no such time series is available.

Now, use of the linear difference \(\frac{S_2 - S_1}{t}\) corresponds to
assumption of a static linear growth rate dynamic, just as use of
\(\frac{log(S_2) - log(S_1)}{t}\) corresponds to assuming a constant
exponential growth rate dynamic. In the latter case, the log-measure has
the nice property of representing an ergodic observable (sense Peters
and Gell-man 2016). While widely (and rightly) dismissed as unrealistic,
the linear dynamic \(\frac{S_2 - S_1}{t}\) may in fact be a generally
superior measure for ecological analysis where no time series of
size/biomass data is available.

As can be seen, the linear model is a far better approximation where
data are taken from within the center part of the growth cycle.

\begin{verbatim}
## Warning: Removed 18 rows containing missing values (geom_path).
\end{verbatim}

\begin{verbatim}
## Warning: Removed 2 rows containing missing values (geom_path).
\end{verbatim}

\includegraphics{Growth_Measure_MS_files/figure-latex/unnamed-chunk-1-1.pdf}

Even where collection of data is exclusively from the convex portion of
the growth curve, the superior fit of the exponential model provides
only a marginal gain in the convex portion of the curve, and then in
absolute value, quickly diverges thereafter.

\begin{Shaded}
\begin{Highlighting}[]
\NormalTok{t2 <-}\StringTok{ }\DecValTok{14}
\NormalTok{t1 <-}\StringTok{ }\DecValTok{2}


\KeywordTok{ggplot}\NormalTok{(}\DataTypeTok{data =} \KeywordTok{data.frame}\NormalTok{(}\DataTypeTok{t=}\KeywordTok{c}\NormalTok{(}\DecValTok{0}\NormalTok{,}\DecValTok{50}\NormalTok{)),}\KeywordTok{aes}\NormalTok{(}\DataTypeTok{x=}\NormalTok{t)) }\OperatorTok{+}\StringTok{ }\KeywordTok{stat_function}\NormalTok{(}\DataTypeTok{fun =}\NormalTok{ sigmoid2, }\DataTypeTok{args =} \KeywordTok{list}\NormalTok{(Po,K,r)) }\OperatorTok{+}\StringTok{ }\KeywordTok{stat_function}\NormalTok{(}\DataTypeTok{fun =}\NormalTok{ sigmoid_exp_approx, }\DataTypeTok{args =} \KeywordTok{list}\NormalTok{(t1,t2,Po,K,r), }\DataTypeTok{color =} \StringTok{"red"}\NormalTok{) }\OperatorTok{+}
\StringTok{  }\KeywordTok{stat_function}\NormalTok{(}\DataTypeTok{fun =}\NormalTok{ sigmoid_lin_approx, }\DataTypeTok{args =} \KeywordTok{list}\NormalTok{(t1,t2,Po,K,r),}\DataTypeTok{color =} \StringTok{"green"}\NormalTok{) }\OperatorTok{+}
\StringTok{  }\KeywordTok{theme_bw}\NormalTok{() }\OperatorTok{+}\StringTok{ }\KeywordTok{scale_y_continuous}\NormalTok{(}\DataTypeTok{limits =} \KeywordTok{c}\NormalTok{(}\DecValTok{0}\NormalTok{,}\DecValTok{15}\NormalTok{))}
\end{Highlighting}
\end{Shaded}

\begin{verbatim}
## Warning: Removed 33 rows containing missing values (geom_path).
\end{verbatim}

\includegraphics{Growth_Measure_MS_files/figure-latex/unnamed-chunk-2-1.pdf}

In summary, the chief virtue of the \(\frac{log(S_2) - log(S_1)}{t}\)
measure is that it effectively linearizes the differences in size from
the convex portion of biological growth curve. Thus, it arguably might
increase the ability to discern subtle but consequential differences in
growth rates in experiments or observations. However, this strength is
also a liability - given process and measurement error, I suspect it
inflates the odds of infering patterns that do not exist.

The much maligned linear measure is a superior default on two grounds
therefore. First, it corresponds far more directly with current
ecological reality. It is a measure with an interpretable biological
dimension (usually mass or length) that helps us understand and describe
our system. Interpreted as a dynamic, it is obviously flawed, but is
demonstrably better in e.g.~MSE than the exponential dynamic.

The widespread use of ``RGR'' \(\frac{log(S_2) - log(S_1)}{t}\) should
be abandoned. Where only two or three time points are available, fitting
a linear growth trend is just as good if not better than estimating an
exponential growth rate. The ideal scenario is to collect a proper time
series (5-7+) and fit a proper growth model. Where data are at all
limiting, we recommend careful incorporation of literature values and
other external information as priors in a fully Bayesian analysis in
order to regularize inferences.


\end{document}
